%%%%%%%%%%%%%%%%%%%%%%%%%%%%%%%%%%%%%%%%%%%%%%%%%%%%%%%%%%%%%%%%%%%%%%%%
%    INSTITUTE OF PHYSICS PUBLISHING                                   %
%                                                                      %
%   `Preparing an article for publication in an Institute of Physics   %
%    Publishing journal using LaTeX'                                   %
%                                                                      %
%    LaTeX source code `ioplau2e.tex' used to generate `author         %
%    guidelines', the documentation explaining and demonstrating use   %
%    of the Institute of Physics Publishing LaTeX preprint files       %
%    `iopart.cls, iopart12.clo and iopart10.clo'.                      %
%                                                                      %
%    `ioplau2e.tex' itself uses LaTeX with `iopart.cls'                %
%                                                                      %
%%%%%%%%%%%%%%%%%%%%%%%%%%%%%%%%%%
%
%
% First we have a character check
%
% ! exclamation mark    " double quote  
% # hash                ` opening quote (grave)
% & ampersand           ' closing quote (acute)
% $ dollar              % percent       
% ( open parenthesis    ) close paren.  
% - hyphen              = equals sign
% | vertical bar        ~ tilde         
% @ at sign             _ underscore
% { open curly brace    } close curly   
% [ open square         ] close square bracket
% + plus sign           ; semi-colon    
% * asterisk            : colon
% < open angle bracket  > close angle   
% , comma               . full stop
% ? question mark       / forward slash 
% \ backslash           ^ circumflex
%
% ABCDEFGHIJKLMNOPQRSTUVWXYZ 
% abcdefghijklmnopqrstuvwxyz 
% 1234567890
%
%%%%%%%%%%%%%%%%%%%%%%%%%%%%%%%%%%%%%%%%%%%%%%%%%%%%%%%%%%%%%%%%%%%
%

%AIP Reprint Class%%%%%%%%%%%%%%%%%%%%%%%%%%%%%%%%%%%%%%%%%%%%%%%%%%%%%%%%%%%%%%%%%%%%%%%%%%%%%
\documentclass[aip,prl,amsmath,amssymb,reprint,superscriptaddress]{revtex4-1} %preprint version
\usepackage{graphicx}% Include figure files
\usepackage{dcolumn}% Align table columns on decimal point
\usepackage{bm}% bold math
\usepackage{epstopdf}

    \renewcommand{\topfraction}{0.9}    % max fraction of floats at top
    \renewcommand{\bottomfraction}{0.8}    % max fraction of floats at bottom
    \setcounter{topnumber}{2}
    \setcounter{bottomnumber}{2}
    \setcounter{totalnumber}{4}     % 2 may work better
    \setcounter{dbltopnumber}{2}    % for 2-column pages
    \renewcommand{\dbltopfraction}{0.9}    % fit big float above 2-col. text
    \renewcommand{\textfraction}{0.07}    % allow minimal text w. figs
    \renewcommand{\floatpagefraction}{0.7}    % require fuller float pages
    \renewcommand{\dblfloatpagefraction}{0.7}    % require fuller float pages
    \setlength{\abovecaptionskip}{5pt}
    \setlength{\belowcaptionskip}{5pt}
    \setlength{\parskip}{0pt}
    \setlength{\textfloatsep}{5pt} 

%%%%%%%%%%%%%%%%%%%%%%%%%%%%%%%%%%%%%%%%%%%%%%%%%%%%%%%%%%%%%%%%%%%%%%%%%%%%%%%%%%%%%%%%%%%%%%%%%%

%IOP preprint class %%%%%%%%%%%%%%%%%%%%%%%%%%%%%%%%%%%%%%%%%%%%%%%%%%%%%%%%%%%%%%%%%%%%%%%%%%%%%%
%\documentclass[12pt]{iopart}
%\newcommand{\gguide}{{\it Preparing graphics for IOP journals}}
%Uncomment next line if AMS fonts required
%\usepackage{iopams}
%\usepackage{graphicx}
%\usepackage{epstopdf}  
%%%%%%%%%%%%%%%%%%%%%%%%%%%%%%%%%%%%%%%%%%%%%%%%%%%%%%%%%%%%%%%%%%%%%%%%%%%%%%%%%%%%%%%%%%%%%%%%%%
%Slava's inserts %%%%%%%%%%%%%%%%%%%%%%%%%%%%%%%%%%%%%%%%%%%%%%%%%%%%%%%%%%%%%%%%%%%%%%%%%%%%%%
%\usepackage{amsfonts}
%\usepackage{amssymb}

%\newcommand{\ptt}[1]{\frac{\partial#1}{\partial t}}
%\newcommand{\vvec}{\mathbf{v}}
%\newcommand{\Bvec}{\mathbf{B}}
%\newcommand{\Evec}{\mathbf{E}}
%\newcommand{\Jvec}{\mathbf{J}}
%\newcommand{\Avec}{\mathbf{A}}
%%%%%%%%%%%%%%%%%%%%%%%%%%%%%%%%%%%%%%%%%%%%%%%%%%%%%%%%%%%%%%%%%%%%%%%%%%%%%%%%%%%%%%%%%%%%%%%%%%

\begin{document}
\title{Variance Anisotropy in a laboratory plasma}

\author{D.A. Schaffner}
\affiliation{Swarthmore College, Swarthmore, PA, USA}
\author{M.R. Brown}
\affiliation{Swarthmore College, Swarthmore, PA, USA}
\author{V.S. Lukin}
\affiliation{Space Science Division, Naval Research Laboratory, Washington, DC, USA}SA}

\date{\today}
\begin{abstract}
Variance anisotropy observed in magnetic fluctuations on the Swarthmore Spheromak Experiment (SSX).
\end{abstract}

\maketitle

\section{Introduction}

The solar wind turbulence community has seen a flurry of results in recent years, as diagnostic capabilities on spacecraft have steadily improved, especially in the temporal resolution which has allowed for investigation into ion, sub-ion and electron scale turbulence measurements. In addition, newer analysis techniques have been developed to better tap the details of the rich turbulent environment that is the heliosphere. In particular, the focus on variance anisotropy---the observation of greater magnetic fluctuation power in components perpendicular to the mean magnetic field versus parallel---has garnered great interest as it is viewed as a keystone result for theoretical understanding of MHD turbulence. Anisotropy is a common theme in solar wind turbulence and is predicted to be present by many turbulence theories. Since the solar wind speeds are typically super-Alfvenic, anisotropy studies generally reference the velocity vector of the bulk flow when defining perpendicular versus parallel fluctuations. Variance anisotropy, however, references a mean magnetic field vector---either global or local mean.

As results from space become more detailed, comparison to simulation and laboratory experiment become more useful in order to better develop the theory as well as inform future space missions. Experiments conducted in the MHD wind-tunnel configuration of the Swarthmore Spheromak Experiment have shown the ability to produce and analyze MHD turbulence using many of the same methods used in the space plasma community, and has produced turbulence measurements which are comparable to \textit{in-situ} results~\cite{schaffner14a}. Moreover, using varying amounts of helicity injection, some characteristics of the turbulence can be modified~\cite{schaffner14b}.

A full spectral analysis of fluctuations in the SSX has been conducted, including magnetic field, density and Mach flow fluctuations. Using a wavelet transform method to decompose the timeseries signal, the magnetic field fluctuation spectra can be broken into portions that are parallel or perpendicular to the local magnetic field vector. Analysis of these portions show that parallel fluctuation power decreases slightly faster than that for perpendicular fluctuations generating a separating in power as a function of increasing frequency. This effect in frequency space is very close to the observation of variance anisotropy seen in wavenumber space in solar wind turbulence. The ratio is shown to grow from nearly isotropic levels in the energy injection scale to a maximum $\perp/\parallel$ ratio of 3 at frequencies of about 1MHz. Beyond this point, the ratio begins to contract reaching isotropy around 10MHz. Though the axial flow speeds of the plasma in the wind-tunnel sub-Mach, if a Taylor hypthesis were still invoked this peak in the ratio would correspond to the ion inertial scale length. Comparison between magnetic flucutation spectra and flow fluctuation spectra are also reported.

The organization of this paper is as follows: First, a brief description of the plasma laboratory is given. Then, an overview of the analysis techniques used to determine perpendicular and parallel fluctuations is provided. The results of this analysis is provided in section 4 including how spectra scaling varies with helicity injection, and how it evolves in time. In section 5 a comparison of magnetic fluctuation spectra is made to velocity and density fluctuations as well a discussion of how all of these spectra results are consistant with the observation of an ion inertial scale in the plasma. Comparison to a MHD simulation is made in section 6 and conclusions are presented in section 7.

%\begin{figure}[!htbp]
%\centerline{
%\includegraphics[width=8.5cm]{helicity_scaling.png}}
%\caption{\label{fig:helicity_scaling} (a) Average magnetic field magnitude over the equilibrium epoch (40-60$\mu s$), (b) injected gun energy and volume integrated magnetic field energy, and (c) amount of helicity injected in the first $20 \mu s$ after discharge trigger as a function of magnetic flux in gun core.}
%\end{figure}

\section{Theory and Techniques}

As noted above, the magnetic correlation function can be written
%
\begin{equation}
R(r) =  \langle {\bf b(x) \cdot b(x + r)} \rangle
\end{equation}
%
where {\bf b} is the fluctuating part of a turbulent magnetic field (${\bf B}(x, t) = {\bf B_0 + b}$)). 

The magnetic Taylor microscale can be formally defined as
%
\begin{equation}
\lambda_T^2 = \frac{\langle b^2 \rangle}{\langle (\nabla \times b)^2 \rangle}.
\label{eq:tayscale}
\end{equation}
%
This definition identifies the Taylor microscale as the scale associated with mean square spatial derivatives of the fluctuating magnetic field $b$.  It is at this scale that one would expect dissipation effects to become important, although actual dissipation likely occurs at smaller, kinetic scales ($k_D \lambda_T = R_m^{1/4}$) \cite{Matthaeus08}.  We expect that the Taylor microscale should be on the same order but larger than the Larmor scale ($\rho_i \approx 1~mm$ in the SSX wind tunnel) or the ion inertial scale ($c/\omega_{pi} \approx 5~mm$ in SSX).  A similar definition of the Taylor microscale in conventional fluids involves spatial derivatives of the fluctuating velocity field \cite{frisch95}.

The correlation scale can be evaluated
%
\begin{equation}
\lambda_{CS}  = \frac{1}{\langle b^2 \rangle} \int_0^\infty R(r) dr
\label{eq:tayscale2}
\end{equation}
%
which is of the order of the system size (eg. the radius of the SSX wind tunnel).

An equivalent formulation of the Taylor scale in terms of the spectrum is
%
\begin{equation}
\frac{1}{\lambda_T^2}  = \int_0^\infty dk k^2 E(k)/\int_0^\infty dk E(k)
\label{eq:tayscale3}
\end{equation}
%
The correlation function can be approximated
\begin{equation}
R(r) \approx  \langle b^2 \rangle \left(1 - \frac{r^2}{2 \lambda_T^2}  \right)
\label{eq:correlation2}  
\end{equation}
%
We will fit our measured 16-point correlation function R(r) to this form in order to extract the Taylor microscale $\lambda_T$. 

An effective turbulent magnetic Reynold's number can be written \cite{Batchelor70}
%
\begin{equation}
R_m^{eff}  = \left(\frac{\lambda_{CS}}{\lambda_T} \right)^2
\label{eq:magReynNum}
\end{equation}
%
Note that the formal definition of the magnetic Reynolds number involves the plasma (Spitzer) conductivity
\begin{equation}
R_m = \mu_0 V \sigma_{SP} L.
\label{eq:magReynNum2} 
\end{equation}
%
Using $T_e = 10~eV$ for the SSX wind tunnel and the radius of the tunnel for the outer scale ($L = R = 0.078~m$), we calculate $R_m = 300$ for a flow speed of 50 km/s.

The Fourier transform of the correlation function is the spatial energy spectrum, $E(k)$:
%
\begin{equation}
E(k) = \frac{1}{2 \pi} \int_0^L e^{ikr} R(r) dr
\label{eq:spectrum1}
\end{equation}
%
We can compare a measurement of the energy spectrum $E(k)$ to the Fourier transform of the measured correlation function $R(r)$.

\section{Experimental Apperatus}

\begin{figure}[!htbp]
\centerline{
\includegraphics[width=8.5cm]{chamber.png}}
\caption{\label{fig:chamber} Cross-section of SSX chamber in MHD wind-tunnel configuration}
\end{figure}

\section{Results}

\begin{figure}[!htbp]
\centerline{
\includegraphics[width=8.5cm]{BrBtBz_correlations.png}}
\caption{\label{fig:BrBtBz_correlations} Correlations Functions}
\end{figure}

\begin{figure}[!htbp]
\centerline{
\includegraphics[width=8.5cm]{figure3_placeholder.png}}
\caption{\label{fig:figure3_placeholder} (a) Single vs Merging. (b) Expt vs Sim.}
\end{figure}

\begin{figure}[!htbp]
\centerline{
\includegraphics[width=8.5cm]{figure4_placeholder.png}}
\caption{\label{fig:figure4_placeholder} $Chi^{2}/n$ of parabolic fit versus n.}
\end{figure}

\begin{table}
\caption{\label{tab:Rms}Comparison of measured/computed Taylor Microscale and magnetic Reynolds numbers for various cases.}
\begin{tabular}{|l|l|l|l|l|l|}
\hline
Condition&$\lambda_{CS}$&$\lambda_{T}$&$R_{m}$ w/Radius&$R_{m}$ w/Int. Scale&$R_{m}$ Computed\\
\hline
Single-Outer&1&1&1&1&1\\
Single-Inner&1&1&1&1&1\\
Merging&1&1&1&1&1\\
Simulation&1&1&1&1&1\\
\hline
\end{tabular}
\end{table}

% ----------------------------------------------------------------
\section*{Acknowledgements}
%  We gratefully acknowledge many useful discussions with William Matthaeus. This work has been funded by the US DoE Experimental Plasma Research program and the National Science Foundation.  The simulations were performed using the advanced computing resources (Cray XC30 Edison system) at the National Energy Research Scientific Computing Center.
% ----------------------------------------------------------------
\section*{References}
\begin{thebibliography}{99}

\bibitem{schaffner14a} D.A. Schaffner {\it et al.} Turbulence analysis of an experimental flux rope plasma. Accepted by Plas. Phys. Cont. Fusion. Dec 2013.

\bibitem{schaffner14b} D.A. Schaffner {\it et al.} Observation of turbulent intermittency scaling with magnetic helicity in an MHD plasma wind-tunnel. Submitted to PRL? Feb 2014.

%\bibitem{Belmabrouk98} Belmabrouk, H., and M. Michard (1998), Taylor length scale measurement by laser Doppler velocimetry, Exp. Fluids, 25, 69Ð76.

%\bibitem{Matthaeus05} Matthaeus, W. H. and Dasso, S. and Weygand, J. M. and Milano, L. J. and Smith, C. W. and Kivelson, M. G., Phys. Rev. Lett. 95, 231101 (2005) Spatial Correlation of Solar-Wind Turbulence from Two-Point Measurements

%\bibitem{Weygand07} Weygand, J. M., Matthaeus, W. H., Dasso, S., Kivelson, M. G., and Walker, R. J. (2007), J. Geophys. Res., 112, A10201.

%\bibitem{Weygand09} Weygand, J. M., Matthaeus, W. H., Dasso, S., Kivelson, M. G., Kristler, L. M., and Mouikis, C. (2009), J. Geophys. Res., 114, A07213.

%\bibitem{Weygand10} Weygand, J. M., Matthaeus, W. H., El-Alaoui, M., Dasso, S., and Kivelson, M. G. (2010), J. Geophys. Res., textit115, A12250.

%\bibitem{Weygand11} Weygand, J. M., Matthaeus, W. H., Dasso, S., and Kivelson, M.G. (2011), J. Geophys. Res., 116, A08120.

%\bibitem{Matthaeus08} Matthaeus W. H., Weygand, J. M., Chuychai, P., Dasso, S., Smith, C. W., and Kivelson, M. (2008), Astrophys. J., 678, L141.

%\bibitem{frisch95}Frisch, U. 1995, {\it Turbulence} (Cambridge: Cambridge Univ. Press)

%\bibitem{sorrisovalvo99}Sorriso-Valvo, L. {\it et al.} Geophys. Res. Lett. {\bf 26}, 1801–1804 (1999).

%\bibitem{wan12}Wan, M. {\it et al.} ApJ. {\bf 744} 177 (2012).

%\bibitem{sorrisovalvo01}Sorriso-Valvo, L. {\it et al.} Planet. Space Sci. {\bf 49}, 1193–1200 (2001).

%\bibitem{marrelli05}Marrelli, L. {\it et al.} Phys. Plasmas. {\bf 12}, 030701 (2005).

%\bibitem{Greco08}A. Greco, P. Chuychai, W. H. Matthaeus, S. Servidio and P. Dmitruk, Intermittent MHD structures and classical discontinuities, Geophys. Res. Lett. {\bf 35}, L19111 (2008).

%\bibitem{Greco09}Greco, A., Matthaeus, W. H., Servidio, S., Chuychai, P., and Dmitruk, P.: Statistical Analysis of Discontinuities in Solar Wind ACE Data and Comparison with Intermittent MHD Turbulence, ApJ {\bf 691}, L111 (2009).

%\bibitem{Wan09}Wan, M., Oughton, S., Servidio, S., and Matthaeus, W. H.: Generation of non-Gaussian statistics and coherent structures in ideal magnetohydrodynamics, Phys. Plasmas {\bf 16}, 080703 (2009).

%\bibitem{Servidio11b}Servidio, S. {\it et al}, J. Geophys. Res. {\bf 116}, A09102 (2011).

%\bibitem{Gray13} T. Gray, M. R. Brown, and D. Dandurand. Phys. Rev. Lett. {\bf 110}, 085002 (2013). 



%\bibitem{Taylor86} J. B. Taylor, Rev. Mod. Phys. {\bf 58}, 741 (1986).

%\bibitem{Matthaeus80} W.H. Matthaeus and D. Montgomery, Ann. N.Y. Acad. Sci. {\bf 357}, 203 (1980).

%\bibitem{torrence98}C. Torrence, G.P. Compo, A practical guide to wavelet analysis. Bull. Am. Meteorol. Soc. {\bf 79}, 6178 (1998).

%\bibitem{clauset09}A. Clauset, C. Rohilla Shalizi, M.E.J. Newman, Power-law distributions in empirical data, SIAM Rev. {\bf 51}, 661703 (2009).

%\bibitem{wan12}M. Wan, K. T. Osman, W. H. Matthaeus, and S. Oughton, Investigation of intermittency in magnetohydrodynamics and solar wind turbulence: scale-dependent kurtosis, ApJ {\bf 744}, 171 (2012).

%\bibitem{Gray10}T. Gray, V. S. Lukin, M. R. Brown, C. D. Cothran, Three-dimensional reconnection and relaxation of merging spheromak plasmas, Phys. Plasmas {\bf 17}, 102106 (2010).

%\bibitem{goldstein94}Goldstein, M.L., Roberts, D.A. and Fitch, C.A. Jour. Geo. Res. {\bf 99} 11519-11538 (1994).

%\bibitem{ji95}Ji, H., Prager, S.C. and Sarff, J.S. Phys. Rev. Lett. {\bf 74} 2945 (1995).

%\bibitem{telloni12}Telloini, D. {\it et al.}. ApJ. {\bf 751} 19 (2012).

%\bibitem{matthaeusVelli11}Matthaeus, W.H. and Velli, M. Space Sci. Rev. {\bf 160} 145-168 (2011).

%\bibitem{greco12}A. Greco {\it et al.} ApJ. {\bf 749} 105 (2012).

\end{thebibliography}

\end{document}

